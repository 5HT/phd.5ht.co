\chapter*{ВСТУП}
\epigraph{Присвячується піонерам формальної школи філософії}
         {Фреге, Расселу, Вайтхеду, Геделю, Гільберту, Каррі, Черчу}

У вступі розказується про новий формальний підхід до математичної верифікації та спробу автора
у цій парадигмі побудувати замкнену уніфіковану систему формальних мов для
програмування, математики та філософії. В процесі розробки моделі такої системи автору
довелося апробувати частини її імплементації для головних SML-подібних формальних академічних мов,
мови Erlang та інших (загалом 7 мов). За 10 років автором було проаналізовані
синтаксис та семантика основних мов програмування (більше 50 мов) з різних промислових
та академічних доменів, 8 мов з яких були особисто реалізовані автором. В роботі
описані 8 мов уніфікованої мовної системи (концептуальна модель) та представлені 2 їх імплементації.

Головним чином, натхнення було почерпнуте з LISP-машин минулого, APL-систем,
перших систем доведення теорем таких як AUTOMATH, віртуальних маших паралельної
та узгодженої обробки нескінченних процесів, таких як BEAM, та кубічних MLTT-пруверів.

\section*{Вступне слово}
Якщо говорити про математичну логіку, формальну математику, формальні методи,
то основоположниками цих теорій можно вважати Бертрана Рассела та Альфреда Норта Вайтхеда та їх роботу
Principia Mathematica\footnote{\url{https://principia.lol} --- імплементація метатеорії Principia Mathematica для Menhir та OCaml},
де будується формально теорія множин та доводиться твердження 1+1=2. Пізніше методи та предмет
лябда-числення був розроблених Хаскелем Каррі та Алонсо Черчем, а теорема Геделя про неповноту до цих
знаходить своє відображення в інфініті-топосах та у зліченних всесвітах сучасних пруверів.

Зараз формальні методи верифікації та відповідно теорії на яких вони побудовані
є актуальними засобами забезпечення математичної якості та гарантій для розробки не тільки
програмного забезпечення, але і математики та навіть формальної філософії.

\section{Актуальність роботи}
Ціна помилок в індустрії надзвичайно велика. Основні відомі приклади в індустрії:
1) Mars Climate Orbiter (1998),
   помилка\footnote{Mars Climate Orbiter Mishap Investigation Board Phase I Report November 10, 1999. \\
                    \url{https://llis.nasa.gov/llis_lib/pdf/1009464main1_0641-mr.pdf}}
   невідповідності типів британської метричної системи, коштувала 80 мільйонів фунтів стерлінгів.
   Невдача стала причиною переходу
   NASA\footnote{National Aeronautics and Space Administration, національна адміністрація аеронавтики та космосу США}
   повністю на метричну систему в 2007 році.
2) Ariane Rocket (1996),
   причина\footnote{ARIANE 5 Flight 501 Failure, \\
          \url{http://www-users.math.umn.edu/~arnold/disasters/ariane5rep.html}}
   катастрофи -- округлення 64-бітного дійсного числа до 16-бітного.
   Втрачені кошти на побудову ракети та запуск 500 мільйонів доларів.
3) Помилка в FPU в перших Pentium (1994), збитки на 300 мільйонів доларів.
4) Помилка\footnote{The Matter of Heartbleed. \\
                    \url{http://mdbailey.ece.illinois.edu/publications/imc14-heartbleed.pdf}}
   в SSL (heartbleed), оцінені збитки у розмірі 400 мільйонів доларів.
5) Помилка у логіці бізнес-контрактів EVM (неконтрольована рекурсія), збитки 50 мільйонів,
   що привело до появи верифікаторів та валідаторів
   контрактів\footnote{Vandal: A Scalable Security Analysis Framework for Smart Contracts, \\
                       \url{https://arxiv.org/pdf/1809.03981.pdf}},
             \footnote{Short Paper: Formal Verification of Smart Contracts, \\
                       \url{https://www.cs.umd.edu/~aseem/solidetherplas.pdf}}.
Більше того, і найголовніше, помилки у програмному забезпеченні можуть коштувати життя людей.

Таким чином зросла популярність формальних мов з типизацією, які дають певні гарантії
на стадії компіляції, витрачаючи при цьому небагато часу на специфікації. Піонер в цьому класі
мов --- Standard ML та ML-подібні мови.

Пізніше, більш загально, Системи Жирара $F$ та $F_\omega$ стали промисловим стандартом де-факто.
Усі сучасні мови для популярних віртуальних машин та середовищ виконання намагаються бути
близькими до цих типових систем. Для JVM існує мова Scala, але перша формальна мова для
JVM був порт Standard ML --- MLj. Для CLR існує мова F\#. Мова Haskell поставляється разом
з середовищем виконання GHC. Усі ці мови так чи інакше пов'язані з системами Жирара.

\section{Формалізована постановка задачі}
Після 5 років цього дослідження, виконавши у вигляді вправ декілька імплементацій мов,
було прийнято рішення формально оформити всю роботу згідно академічних нормативів.
Тому в цій секції дається формальна постановка задачі, яка складається з опису
об'єкту та предмету дослідження, мети, цілей та завдань.
Дається головна мотивація та аналіз результатів.

\subsection{Мотивація та мета дослідження}
Головна мотивація цієї роботи --- пошук єдиної мови або мовної системи, що здатна стати єдиною мовою,
яка пропонує аналогічний спрощений формальний спосіб програмування та доведення теорем.

Одна з причин низького рівня впровадження у виробництво систем
верифікації -- це висока складність таких систем. Складні системи
верифікуються складно. Ми хочемо запропонувати спрощений
підхід до верифікації --- оснований на концепції компактних
та простих мовних ядер для створення специфікацій, моделей,
перевірки моделей, доведення теорем у теорії типів з кванторами.

Метою цього дослідження є побудова єдиної системи, яка поєднує формальне середовище
виконання та систему верифікації програмного забезпечення з великим спектром мовних
засобів, які допомагають вбудувати в себе максимальну кількість існуючих мов.
Це прикладне дослідження, яке є сплавом фундаментальної математики та інженерних
систем з формальними методами верицікації.

\subsection{Цілі та завдання дослідження}
Головними цілями цього дослідження є побудова мінімальної системи
мовних засобів для побудови ефективного циклу верифікації програмного
забезпечення та доведення теорем. Основні компоненти системи, як продукт дослідження:
1) інтерпретатор безтипового лямбда числення;
2) компактне ядро --- система з однією аксіомою;
3) мова з індуктивними типами;
4) мова з гомотопічним інтервалом $[0,1]$;
5) уніфікована базова бібліотека;
6) бібліотека математичних компонент.

Завданням цього дослідження є імплементація (апробація)
специфікації системи мов на різних мовах програмування та типових системах.
Для цього детально проводився аналіз усіх існуючих мов
програмування (близько 2000), які детально прокатегоризовані
в Енциклопедії Мов Програмування.

\subsection{Об'єкт та предмет дослідження}
Об'єктом дослідження в широкому сенсі є множина
всіх формальних мов, та можливих зв'язків між ними.

Формально, об'єктом дослідження данної роботи є:
1) системи верифікації програмного забезпечення;
2) системи доведення теорем;
3) мови програмування;
4) операційні системи, які виконують обчислення в реальному часі;
3) їх поєднання, побудова формальної системи для
уніфікованого середовища, яке поєднує середовище
виконання та систему верифікації у єдину систему мов та засобів.

\subsection*{Лямбда куб Барендрехта}

\begin{center}
\begin{tikzpicture}
  \matrix (m) [matrix of math nodes, row sep=3em,
    column sep=3em]{
    & \lambda_\omega & & \lambda P_\omega \\
    \lambda_2 & & \lambda P_2 & \\
    & \lambda_{\underline{\omega}} & & \lambda P_{\underline{\omega}} \\
    \lambda_\rightarrow & & \lambda P \\};
  \path[-stealth]
    (m-1-2) edge (m-1-4) edge (m-2-1)
            edge [densely dotted] (m-3-2)
    (m-1-4) edge (m-3-4) edge (m-2-3)
    (m-2-1) edge [-,line width=6pt,draw=white] (m-2-3)
            edge (m-2-3) edge (m-4-1)
    (m-3-2) edge [densely dotted] (m-3-4)
            edge [densely dotted] (m-4-1)
    (m-4-1) edge (m-4-3)
    (m-3-4) edge (m-4-3)
    (m-2-3) edge [-,line width=6pt,draw=white] (m-4-3)
            edge (m-4-3);
\end{tikzpicture}
\end{center}

Предметом та методом дослідження такої системи мов є теорія типів,
як сучасний фундамент математики,
який стисло та компактно представляє не тільки теорію множин,
але і теорію категорій, алгебраїчну топологію та дифференціальну геометрію.
Теорія типів вивчає обчислювальні властивості мов та виділилася
в окрему науку Пером Мартіном-Льофом як запит на вакантне місце у
трикутнику теорій, які відповідають ізоморфізму
Каррі-Говарда-Ламбека (Логіки, Мови, Категорії).

\subsection{Методи дослідження}
Існує багадо підходів для формальної специфікації,
верифікації та валідації, усі вони даються у розділі 1 та
бгрунтувується вибір методу моделювання з використанням
мови з залежними типами (теорії типів Мартіна-Льофа).
Для разкриття семантки цього методу використовується
категорний метод та категоріальна логіка -- теорія топосів.
Теорія категорій довела свою корисність не тільки для
математики\footnote{Близько 10 робіт медалістів премії
Філдса грунтуються на категорних методах},
але і для програмного
забезпечення\footnote{Категорні бібліотеки таких мов як Haskell, Scala тощо.}

\subsection{Класифікація мов програмування}
Повна класифікація об'єктів дослідження була злоблена в рамках проекту
Енциклопедія Мов Програмування, де була зроблена спробо надати
формальну БНФ-нотацію для тих мов, для яких це не було ще ніколи
не зроблено (наприклад для APL-подібної мови K).

\begin{table}
  \caption{Класифікація мов програмування}
 \begin{tabular}{lcc}
    \hline
       \textbf{Домен} & \textbf{Мови програмування} \\
    \hline
       HW & VHDL, Verilog, Clash, Chisel, SystemC, Lava, BSV \\
    \hline
\rowcolor{LightGray25}
       ASM & PDP-11, VAX, S/360, M68K,  \\
\rowcolor{LightGray25}
           & PowerPC, MIPS, SPARC, Super-H \\
\rowcolor{LightGray25}
           & Intel, ARM, RISC-V \\
    \hline
       ALG & C, BCPL, ALGOL, SNOBOL, Simula, \\
           & Pascal, Oberon, COBOL, PL/1 \\
    \hline
\rowcolor{LightGray}
       ML & SML, Alice ML, OCaml, UrWeb, Flow, F\# \\
    \hline
\rowcolor{LightGray}
       PURE & HOPE, Miranda, Clean, Charity, Joy, Mercury, Elm, PureScript \\
    \hline
\rowcolor{LightGray}
       F$_\omega$ & Scala, Haskell, 1ML, Plutus \\
    \hline
\rowcolor{LightGray25}
       MACR & LISP, Scheme, Clojure, Racket, Dylan, LFE, CL \\
\rowcolor{LightGray25}
            & Nemerle, Nim, Haxe, Perl, Elixir \\
    \hline
\rowcolor{LightGray25}
       OOI & Simula, Smalltalk, Self, REBOL, Io \\
\rowcolor{LightGray25}
           & JS, Lua, Ruby, Python, PHP, TS, Java, Kotlin \\
    \hline
\rowcolor{LightGray25}
       CMP & C++, Rust, D, Swift, Fortran \\
    \hline
       SHELL & PowerShell, TCL, SH, CLIPS, BASIC, FORTH \\
    \hline
       SVC & IDL, SOAP, ASN.1, GRPC \\
    \hline
       MARK & TeX, PS, XML, SVG, CSS, ROFF, OWL, SGML, RDF, SysML \\
    \hline
       LOGIC & AUT-68, ACL2, LEGO, ALF, Prolog \\
             & CPL, Mizar, Dedukti, HOL, Isabelle, Z \\
    \hline
\rowcolor{LightGray}
       $\Pi\Sigma$ & Coq, F*, Lean, NuPRL, ATS, Epigram, \\
\rowcolor{LightGray}
          & Cayenne, Idris, Dhall, Cedile, Kind \\
    \hline
\rowcolor{LightGray}
       HoTT & Menkar, Cubical, yacctt, redtt, RedPRL, Arend, Agda \\
    \hline
\rowcolor{LightGray25}
       CHKR & TLA+, Twelf, Promela, CSPM \\
    \hline
\rowcolor{LightGray}
       PAR & Ling, Pony, Erlang, BPMN, Ada, E, Go, Occam, Oz \\
    \hline
\rowcolor{LightGray}
       ARR & Julia, Wolfram, MATHLAB, Octave, Futhark, APL \\
\rowcolor{LightGray}
           & SQL, cg, Clarion, Clipper, QCL, K, MUMPS, Q, R, S, J, O \\
    \hline
  \end{tabular}
\small Сірим кольором показані фокусні домени дослідження.
\end{table}



\section{Наукова новизна}
Автор спробував поглянути на проблему розширення, або мовного доповнення
топосів та категорій де відбуваються обчислення, з точки зору теорії типів,
та їх мовних синтаксисів та категорій. За допомогою спектрального розкладення на
елементарні мови, які репрезентують певні типи та описуються сигнатурами ізоморфізмів,
будується єдиний погляд на еволюцію мови та її покомпонентний аналіз. Також автор
зазирнув у спектр мов, які доречно використовувати та тримати як мови нижнього
рівня (системне програмування) для програмування середовища виконання.

\paragraph{}
Кожен інститут чи команія інвестує в одну певну мову, для зосередження зусиль на одному проекті.
Особливість цієї роботи полягає в побудові уніфікованої системи, яка системно підходить
до вирішення проблеми розширення мовних ядер, та їх евалуаторів. Ця робота
пропонує замкнений фреймворк, який складається з мінімальної системи мов,
що покриваються максимальну кількість мовних синтаксисів та семантик.

\begin{table}
 \caption{Ландшафт атаки}
  \begin{tabular}{lccccc}
    \hline
       \textbf{Термінальна мова/Ефекти} & \textbf{Чиста мова} & \textbf{Система F} & \textbf{Runtime} & \textbf{Віртуалізація} \\
    \hline
 Agda/MAlonzo      & Morte     & Haskell & GHC    & HaLVM \\
 Coq/coq.io        & N/A [CoC] & OCaml   & native & Mirage \\
 Anders/IOI        & CoC       & OCaml   & native & Mirage \\
    \hline
    \rowcolor{LightGray}
 HTS/IOI           & PTS       & Hamler  & BEAM   & LING \\
    \hline
    \rowcolor{LightGray}
 HTS/IOI           & PTS       & ITS     & CPS    & L4 \\
    \hline
  \end{tabular}
  \small Сірим кольором показані напрямки атаки.
\end{table}

Крім того, попередні дослідники зосереджувались на побудові системних бібліотек для
певного середовиша виконання та асоційованих з ним вищих мов (з відповідними біндінгами).
На відміну від фокусних досліджень, ця робота пропонує мультидисциплінарний або мовний підхід,
де ми зосереджуємося на побудові моделі, яка буде вбудовуватися з мінімальними
зусиллями в основні аглебраїчні мови. У таблиці яка показує можливий ландшафт атаки
мовних систем які можна вважати аналогічними підходами до побудови не тільки замкненого
життєвого циклу мовного забезпечення, але і системи віртуалізації.

Інновація роботи полягає в побудові унікальної замкненої системи яка складається з:
1) системного програмного забезпечення -- модального середовища виконання разом з інтерпретатором
написаним на формальній мові, разом з базовою бібліотекою та архітектурою прикладного програмування N2O.TECH;
2) прикладного програмного забезпечення --- системи вищих формальних мов, для яких надано моделі,
імплементації та базова бібліотека разом з математичними компонентами.

У цій роботі представлені дві конфіурації мовних систем, та три вектора атаки для їх дослідження.
Перша атака --- це побудова замкненої системи мов для компіляції в середовище виконання віртуальної машини BEAM що входить до складу Erlang/OTP.
Друга атака --- це побудова власної віртуальної машини CPS, яка пропонує більш формальну та сучасну модель обчислень.
Третя атака --- це розробка бібліотеки вищих мов що могла би компілюватися в BEAM та/або CPS.\\

\subsection{Формальне середовище виконання}
В рамках розробленого фреймворку будується
архітектура системи доведення теорем та обчислювального
середовища (яке складається з верифікованого засобами Rust
інтерпретатора та операційної системи, подібно до Erlang, але без надлишкових копіювань),
яке побудовано за сучасними стандартами:
1) відсутність системи управління пам'яті у реальному часі (тільки на стадії компіляції);
2) автоматична векторизація за допомогою AVX інструкцій (тензорне ядро);
3) дані ніколи не копіюються;
4) в системі немає очікування та даремного витрачання ресурсів;
5) лінисий лямбда-інтерпретатор, програми якого поміщаються в L1 кеш процесора.

Також додається модель базової бібліотеки середовища виконання для мов Erlang та Hamler,
яка пройшла апробацію на підприємствах.

\subsection{Гомотопічна система верифікації}
Чиста мова з однією аксомою дається як перша мова системи вищих мов.
Далі надається базова бібліотека для вищих мов, яка сумісна з кубічним верифікатором,
в якому аксіома унівалентності Воєводського має конструктивну інтерпретацію,
це cubicaltt авторства Андерса Мортберга (2017). Також надається спектр вищих формальних мов.

Гомотопічні системи типів та системи типів в зв'язаних топосах є сучасним поглядом на
конструктивну математику на шляху до формалізації таких моделей як інфінітезімальні околи,
нескінченно малі (які потрібні для моделювання диференціальної геометрії), тощо.

\section{Практичні результати}
В рамках цього дослідження були досягнуті наступні проміжні практичні результати:
1) проаналізовані та класифіковані всі мови програмування, оформлені у вигляді
   Енциклопедії Мов Програмування\footnote{\url{https://groupoid.github.io/languages}},
   що дозволяють виконати проміжні цілі та головне завдання дослідження --- побудова
   мінімальної системи мов для побудови уніфікованої системи для програмування та доведення теорем,
   яка складається з системи мов середовища виконання та системи вищих мов;
2) розроблений прототип середовища виконання CPS який підтримує імутабельні черги
   та AMP/SMP планування, містить лінивий інтерпретатор, та підтрмує числення процесів, реалізовано на мові Rust з лінійними типами,
   для якої існує формальна модель\footnote{\url{https://arxiv.org/pdf/1804.07608.pdf}}
3) розроблена вища мова програмування PTS з екстрактом в байт-код віртуальної машини BEAM, яка пройла peer review;
4) розроблена базова бібліотека N2O.TECH середовища виконання для віртуальної машини BEAM зі специфікацією в мові ITS;
5) розроблена базова бібліотека та бібліотека математичних компонент для вищої мови програмування HTS.

\subsection{Основні результати}
Автор особисто створив моделі та імплементації формального середовища виконання,
бібліотеки середовища виконання, декількох верифікаторів та базової бібліотеки
як приклади використання та моделювання системи доведення теорем. Автор також
розробив курс гомотопічної теорії типів, на якому здійнюється формалізація
певних розділів математики (теорія категорій, різні частини алгебраїчної
топології та диференціальної геометрії).

Окрім того створений сайт, присвячений документації по базовій бібліотеці
середовища виконання \footnote{\url{https://n2o.tech/ua}},
та по бібліотеці системи вищих мов для кубічного верифікатора гомотопічної
мови програмування \footnote{\url{https://groupoid.space/math}}. Також частина
моделей розроблений автором попала в апстрім кубічного верифікатора, як приклади використання.

\subsection{Продукт дослідження}
Не зважаючи на нетривіальну структуру результатів, головним продуктом цього
дослідження слід вважати загальний підхід описаний в цій роботій, який можна
звести до наступних рекомендацій:
1) побудувати лінивий інтерпретатор нетипизованого лямбда числення разом з системою процесів та чергами повідомлень на формальній мові;
2) побудувати мову числення конструкцій на формальній мові з екстрактом у лінийвий інтерпретатор;
3) побудувати мову системи F для розробки базової бібліотеки для лінивого інтерпретатора;
4) побудувати базову бібліотеку для лінивого інтерпретатора;
5) побудувати гомотопічну мову програмування з екстратом в мову системи F;
6) побудувати базову бібліотеку для гомотопічної мови;
7) побудувати бібліотеку теорем формальної математики.

Цей підхід закріплюється формальною моделлю даною у розділі 2 та далі розвивається у наступних розділах.
Таким чином, можна говорити, що продукт цього дослідження --- це формальна специфікація на систему мов та
трансформації між ними, яка виявилася цікава не тільки автору дослідження.

\subsection{Апробація роботи}
Апробації:
1) Відбулися виступи на двох міжнародних конференціях: MMCTSE, IAI;
2) Впровадження базової бібліотеки середовища виконання в державних компаніях: ПриватБанк, ІНФОТЕХ;
3) Впровадження прототипу середовища виконання описаного в роботі, який ліг в основу комерційного продукту.

\subsection{Аналіз результатів}


\section{Структура роботи та публікації}
Якшо коротко суть роботи зводиться до побудови системи, яка складається з:
i) середовища виконання; ii) формального інтерпретатора; iii) системи формальних мов
для доведення теорем математики, програмної інженерії та філософії.

\subsection*{Формальна верифікація}
У розділі 1 дається огляд існуючих рішень для доведення
властивостей систем та моделей, класифікуються мови програмування
та системи доведення теорем.

\subsection*{Концептуальна модель}
У розділі 2 розглядається математична модель формальної системи,
яка умовно розділяється на систему мов для системного
програмування (спектр мов середовище виконання) та систему мов для
прикладного та програмування вищих логік (спектр вищих мов).
Час дослідження цього направлення припав на 2017-2018 роки.

\subsection*{Формальне середовище виконання}
У розділі 3 разказується про повний стек формального програмного забезпечення
від віртуальної машини, байт-код інтерпретатора, середовища виконання
та планування процесів до формальної мови для доведення теорем (або сімейства мов).
Час дослідження цього направлення припав на 2016-2017 роки.

У розділі 4 описується базова бібліотека системи середовища виконання,
написана на мові нижчого рівня Erlang.
Час дослідження цього направлення припав на 2013-2021 роки.

\subsection*{Система верифікації}
У розділі 5 подається опис системи вищих формальних мов для доведення теорем до
кубічної теорії та гомотопічної системи типів.
Час дослідження цього направлення припав на 2021-2023 роки.

У розділі 6 описується базова бібліотека системи формальних мов, індуктивна
та гомотопічні системи типів для кубічних тайпчекерів.
Час дослідження цього направлення припав на 2018-2021 роки.

\subsection*{Математичні компоненти}
У розділі 7 пропонується ряд математичних моделей та теорій з використанням
базової бібліотеки розділу 3 та мови гомотопічної системи типів.
Час дослідження цього направлення припав на 2018-2021 роки.

\subsection*{Додаткові матеріали}
У додатках надаються приклади іншого використання фомальних мов та моделей,
зокрема для мінімальної формальної мови, побудованої в рамках дисертації,
та мови програмування Coq. А також дається приклад використання
гомотопічної мови для формальної філософії.

\section{Подяка}
У вступі хотів би висловити подяку:
1) двом вчителям старших класів: геометру Панькову та алгебраїсту Конету;
2) тьоем вчителям в університеті: Клименко, Таран та Маслянко;
3) двом анонімним друзям в інтернеті: Adam та Siegmentation Fault;
4) двом комюніті які мене радо прийняли: Lean та HoTT комюніті;
5) усім контрібюторам N2O.TECH\footnote{\url{https://n2o.tech}} та Groupoid Infinity\footnote{\url{https://groupoid.space}};
6) батькам;
7) іншим вчителям.
