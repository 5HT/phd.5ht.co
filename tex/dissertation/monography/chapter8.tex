\chapter{ДОДАТКИ}

У додатках ми використаємо три різних мови, та покаже два застосування
формальних мов: 1) формальна філософія (на мові $O_{HTS}$; 2) формальний ввід-вивід для
системної інженерії (на двох промислових мовах Erlang та OCaml).

\section{Формалізація вводу-виводу для OCaml}

\begin{lstlisting}
CoInductive Co (E : Effect.t) : Type -> Type :=
    | Bind : forall (A B : Type), Co E A -> (A -> Co E B) -> Co E B
    | Split : forall (A : Type), Co E A -> Co E A -> Co E A
    | Join : forall (A B : Type), Co E A -> Co E B -> Co E (A * B).
    | Ret : forall (A : Type) (x : A), Co E A
    | Call : forall (command : Effect.command E),
                    Co E (Effect.answer E command)

Definition run (argv : list LString.t): Co effect unit :=
    ido! log (LString.s "What is your name?") in
    ilet! name := read_line in
    match name with
      | None => ret tt
      | Some name => log (LString.s "Hello " ++ name ++ LString.s "!")
    end.

Parameter infinity : nat.
Definition eval {A} (x : Co effect A) : Lwt.t A := eval_aux infinity x.
\end{lstlisting}

\newpage
\begin{lstlisting}
Fixpoint eval_aux {A} (steps : nat) (x : Co effect A) : Lwt.t A :=
    match steps with
      | O => error tt
      | S steps =>
        match x with
          | Bind _ _ x f => Lwt.bind   (eval_aux steps x)
                            (fun v_x => eval_aux steps (f v_x))
          | Split _ x y  => Lwt.choose (eval_aux steps x)
                                       (eval_aux steps y)
          | Join _ _ x y => Lwt.join   (eval_aux steps x)
                                       (eval_aux steps y)
          | Ret _ v => Lwt.ret v
          | Call c => eval_command c
        end
    end.
\end{lstlisting}

\begin{lstlisting}
CoFixpoint handle_commands : Co effect unit :=
    ilet! name := read_line in
    match name with
      | None => ret tt
      | Some command =>
        ilet! result := log (LString.s "Input: "
                     ++ command ++ LString.s ".")
     in handle_commands
    end.

Definition launch (m: list LString.t -> Co effect unit): unit :=
    let argv := List.map String.to_lstring Sys.argv in
    Lwt.launch (eval (m argv)).

Definition corun (argv: list LString.t): Co effect unit :=
    handle_commands.

Definition main := launch corun.
\end{lstlisting}

\newpage
\section{Формалізація вводу-виводу для Erlang}

This work is expected to compile to a limited number of target platforms.
For now, Erlang, Haskell, and LLVM are awaiting.
Erlang version is expected to be used both on LING and BEAM Erlang virtual machines.
This language allows you to define trusted operations in System F and extract this routine to Erlang/OTP platform and plug as trusted resources.
As the example, we also provide infinite coinductive process creation and inductive shell that linked to Erlang/OTP IO functions directly.

{\bf IO} protocol.
We can construct in pure type system the state machine based on (co)free monads driven by {\bf IO/IOI} protocols.
Assume that {\bf String} is a {\bf List\ Nat} (as it is in Erlang natively), and three external constructors: getLine, putLine and pure.
We need to put correspondent implementations on host platform as parameters to perform the actual IO.

\begin{lstlisting}
String: Type = List Nat
data IO: Type =
     (getLine: (String -> IO) -> IO)
     (putLine: String -> IO)
     (pure: () -> IO)
\end{lstlisting}

\newpage
\subsubsection{Infinity I/O Type}

Infinity I/O Type Spec.

\begin{lstlisting}
-- IOI/@: (r: U) [x: U] [[s: U] -> s -> [s -> #IOI/F r s] -> x] x
   \ (r : *)
-> \/ (x : *)
-> (\/ (s : *)
   -> s
   -> (s -> #IOI/F r s)
   -> x)
-> x

-- IOI/F
   \ (a : *)
-> \ (State : *)
-> \/ (IOF : *)
-> \/ (PutLine_ : #IOI/data -> State -> IOF)
-> \/ (GetLine_ : (#IOI/data -> State) -> IOF)
-> \/ (Pure_ : a -> IOF)
-> IOF

-- IOI/MkIO
   \ (r : *)
-> \ (s : *)
-> \ (seed : s)
-> \ (step : s -> #IOI/F r s)
-> \ (x : *)
-> \ (k : forall (s : *) -> s -> (s -> #IOI/F r s) -> x)
-> k s seed step

-- IOI/data
#List/@ #Nat/@
\end{lstlisting}

\newpage
Infinite I/O Sample Program.

\begin{lstlisting}[mathescape=true]
-- Morte/corecursive
( \ (r: *1)
 -> ( (((#IOI/MkIO r) (#Maybe/@ #IOI/data)) (#Maybe/Nothing #IOI/data))
    ( \ (m: (#Maybe/@ #IOI/data))
     -> (((((#Maybe/maybe #IOI/data) m) ((#IOI/F r) (#Maybe/@ #IOI/data)))
           ( \ (str: #IOI/data)
            -> ((((#IOI/putLine r) (#Maybe/@ #IOI/data)) str)
                (#Maybe/Nothing #IOI/data))))
         (((#IOI/getLine r) (#Maybe/@ #IOI/data))
          (#Maybe/Just #IOI/data))))))
\end{lstlisting}

\newpage
Erlang Coinductive Bindings.

\begin{lstlisting}[mathescape=true]
copure() ->
    fun (_) -> fun (IO) -> IO end end.

cogetLine() ->
    fun(IO) -> fun(_) ->
        L = ch:list(io:get_line("> ")),
        ch:ap(IO,[L]) end end.

coputLine() ->
    fun (S) -> fun(IO) ->
        X = ch:unlist(S),
        io:put_chars(": "++X),
        case X of "0\n" -> list([]);
                      _ -> corec() end end end.

corec() ->
    ap('Morte':corecursive(),
        [copure(),cogetLine(),coputLine(),copure(),list([])]).
\end{lstlisting}

\begin{lstlisting}[mathescape=true]

> om_extract:extract("priv/normal/IOI").
ok
> Active: module loaded: {reloaded,'IOI'}
\end{lstlisting}

\begin{lstlisting}[mathescape=true]
> om:corec().
> 1
: 1
> 0
: 0
#Fun<List.3.113171260>
\end{lstlisting}

\newpage
\subsubsection{I/O Type}

I/O Type Spec.

\begin{lstlisting}[mathescape=true]
-- IO/@
   \ (a : *)
-> \/ (IO : *)
-> \/ (GetLine_ : (#IO/data -> IO) -> IO)
-> \/ (PutLine_ : #IO/data -> IO -> IO)
-> \/ (Pure_ : a -> IO)
-> IO

-- IO/replicateM
   \ (n: #Nat/@)
-> \ (io: #IO/@ #Unit/@)
-> #Nat/fold n (#IO/@ #Unit/@)
               (#IO/[>>] io)
               (#IO/pure #Unit/@ #Unit/Make)
\end{lstlisting}

Guarded Recursion I/O Sample Program.

\begin{lstlisting}[mathescape=true]
-- Morte/recursive
((#IO/replicateM #Nat/Five)
 ((((#IO/[>>=] #IO/data) #Unit/@) #IO/getLine) #IO/putLine))
\end{lstlisting}

\newpage
Erlang Inductive Bindings.

\begin{lstlisting}[mathescape=true]
pure() ->
    fun(IO) -> IO end.

getLine() ->
    fun(IO) -> fun(_) ->
        L = ch:list(io:get_line("> ")),
        ch:ap(IO,[L]) end end.

putLine() ->
    fun (S) -> fun(IO) ->
        io:put_chars(": "++ch:unlist(S)),
        ch:ap(IO,[S]) end end.

rec() ->
    ap('Morte':recursive(),
        [getLine(),putLine(),pure(),list([])]).
\end{lstlisting}


Here is example of Erlang/OTP shell running recursive example.

\begin{lstlisting}[mathescape=true]
> om:rec().
> 1
: 1
> 2
: 2
> 3
: 3
> 4
: 4
> 5
: 5
#Fun<List.28.113171260>
\end{lstlisting}

\newpage
\section{Формалізація мадх'яміки}

Сейчас я дам вам почувствовать вкус формальной философии по-настоящему!
А то, вам может показаться, что это канал по формальной математике,
а не формальной философии. Я же считаю, что если формальная философия
не опирается на формальную математику, то грош цена такой формальной философии. 

\begin{lstlisting}
module buddhism where
import path
\end{lstlisting}

Сегодня мы будем формализировать понятие недвойственности в буддизме,
которое связано сразу со многими концепциями на уровнях Сутры,
Тантры и Дзогчена: понятием взаимозависимого возникновения и
понятием пустотности всех феноменов (Сутра Праджняпарамиты).
Классический пример с расчленением тела ставит вопрос, когда
тело перестает быть человеком-существом, если от него начать
отрубывать куски мяса (мы буддисты любим и лилеем такие мысленные
образы-эксперименты) или другими словами, чтобы отличить тело от
не-тела, нам нужен двуместный предикат (семья типов), функция,
которая может идентифицировать конректные два эклемпляры тела.
Фактически речь идет об индентификации двух объектов, т.е. обычном
типе-равенстве Мартина-Лёфа.

За фреймворк возьмем концепты Готтлоба Фреге, согласно определению,
концепт -- это предикат над объектом или, другими словами, Пи-тип
Мартина-Лёфа, индексированный тип, семья типов, тривиальное
расслоение и т.д. Где объект x из o принадлежит концепту,
только если сам концепт, параметризированный этим объектом,
населен p(o) : U (где p : concept o).

\begin{lstlisting}
concept (o: U): U
  = o -> U
\end{lstlisting}

Концепт p должен предоставлять пример или контрпример в различении,
т.е. чтобы определить тело это или не-тело еще, пока мы его расчленяем,
нам нужно как минимум два куска: тело и не-тело как примеры идентификации.
Таким образом, недвойственность может быть представлена, как равенство
между всеми расслоениями (предекатами над объектами).

\begin{lstlisting}
nondual (o: U) (p: concept o): U
  = (x y: o) -> Path U (p x) (p y)
\end{lstlisting}

Итак, недвойственность устраняет различие между примерами
и контрпримерами на примордиальном уровне мандалы MLTT,
иными словами идентифицирует все концепты. Сама же идентификация
классов объектов, которые принадлежат разным концептам — это условие,
сжимающее все объекты в точку, или стягиваемое пространство,
вершина конуса мандалы MLTT, или, другими словами,
пустотность всех феноменов.

\begin{lstlisting}
allpaths (o: U): U
  = (x y: o) -> Path o x y
\end{lstlisting}

Формулировка буддийской теоремы недвойственности которая
распространяется для всех типов учеников (тупых, средних и смышленых),
может звучать так: недвойственность концепта есть способ идентификации
его объектов. Сформулируем эту же теорему в другую сторону:
способ идентификации объектов задает предикат недвойственности
концептов. Туда — ((p: concept o) -> nondual o p) -> allpaths o,
Сюда — allpaths o -> ((p: concept o) -> nondual o p).
И докажем ее! Как видно из сигнатур нам всего-лишь надо
построить функцию транспорта между двумя пространствами
путей: (p x) =$_U$ (p y) и x =$_o$ y. Воспользуемся приведением
пути в стрелку (coerce) и конгруэнтностью (cong) из базовой библиотеки.

\begin{lstlisting}
encode (o:U): ((p: concept o) -> nondual o p) -> allpaths o
  = \(nd: (p: concept o) -> nondual o p) (a b: o)
  -> coerce(Path o a a)(Path o a b)(nd(\(z:o)->Path o a z)a b)(refl o a)
\end{lstlisting}

\begin{lstlisting}
decode (o:U): allpaths o -> ((p: concept o) -> nondual o p)
  = \(all: allpaths o)(p: concept o)(x y: o) -> cong o U p x y (all x y)
\end{lstlisting}

Как видите, теоремка о пустотности всех феноменов получилась на пару
строчек, которые демонстрируют: 1) основы формальной философии и быстрый вкат;
2) хороший пример к первой главе HoTT на пространство путей и модуль path.

