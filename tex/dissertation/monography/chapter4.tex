\chapter{Базова бібліотека середовища виконання}
\epigraph{Присвячується автору System F}{Жану-Іву Жирару}

\section{Філософія N2O.TECH}
Після побудови в розділі 3 формального середовища виконання,
яке складається з операційної системи у якій виконуються
CPS-інтерпретатори з формальною системою вводу-виводу IO, можна зразу
переходити до базової бібліотеки середовища виконання.

Даний розділ формалізує інтерфейс прикладного програмування та систему
бібліотек часу виконання для забезпечення потреб побудови гетерогенних систем та сервісів.

N2O.TECH визначає специфікацію на ПЗ усіх рівнів прикладної моделі
для підприємств на функціональних мовах програмування. Ця
специфікація визначає правила побудови WebSocket сервера,
бінарного серіалізатора та веб-фреймворку визначеному
формальними протоколами. Промислові версії також підтримують
систему управління бізнес-процесами та ефективне сховище
з глобальним простором ключів.

N2O.TECH — це формальна філософія та інженерна вправа водночас.
Вона обмежує автора бути ефективним та точним не втрачаючи
при цьому повноти та функціональності. Це накшталт внутрішньої
дисципліні при проектуванні програмного забезпечення. Ця філософія
багато років застосовується на практиці для побудови систем SYNRC,
та визначає стандартний мінімальний набір для демонстрації однієї
з сучасних моделей реактивного веб-програмування, яка включає: веб-сокет
веб-фреймвок з бінарною серілізацією, пушами та контролем DOM елементів
зі сторони сервера. N2O.TECH вчить будувати прості та надійні системи
на будь-якій мові програмування.

Філософія N2O.TECH визначає структуру операційних середовищ (runtimes) як
операційну систему лямбда-інтерпретаторів які працюють на паралельному
обчислювальному середовищі (ядрах процесорів). Кожне з ядер процесорів
виконує в нескінченному циклі команди лямбда-інтерпретаторів, переключаючи
через певний проміжок часу на потік команд іншого інтерпретатора. Таке
визначення дає змогу вбудувати цю структуру у віртуальну машину Erlang:
1) Головний процес додатку; 2) Супервізор додатку; 3) Проміжні
супервізори; 4) Кінцеві пул процесорів повідомлень.

\subsection{Серверні та клієнтські мови}
Мови програмування розділені на чотири рівня як для клієнтської
(мобільної та веб розробки) так і для серверної розробки (бекенд
системи). Нульовий рівень — тотальні формальні алгебраїчні мови
програмування, що забезбечують повноту, функціональність та доведення
властивостей прогрм згідно сучасних уявлень про математичне моделювання
та системи залежних типів побудованих на розшаруваннях. Перший
рівень — формальні функціональні мови програмування, як правило
System F, System Fω які успішно використовуються в промисловості
та забезпечують достатньо формальний запис який піддається
масштабуванню у великих командах завдяки потужному ядру компіляторів.
Другий рівень — неформальні (без формальної операційної чи денотаційної
семантики) чи формальних верифікаторів, які проте успішно
використовуються в промисловості, можуть бути як з розвиненими
системами типів з узагальненими шаблонами та типами сумами,
так і однотипними мовами програмування з динамічною типізацією.
Третій рівень — мови які погано піддаються масштабуванню у
промисловому виробництві (на основі спостережень за власним досвідом).

\begin{lstlisting}
    Client Tier 3: (JavaScript, Lua)
    Client Tier 2: (Swift, Kotlin, TypeScript)
    Client Tier 1: (UrWeb, OCaml, PureScript)
    Client Tier 0: (Formality, PTS)

    Server Tier 3: (PHP, Python, Perl, Ruby)
    Server Tier 2: (Erlang, Elixir)
    Server Tier 1: (Standard ML, Haskell, F#, Rust, Hamler)
    Server Tier 0: (Coq, Agda, Lean, MLTT/HTS)
\end{lstlisting}

\subsection{Обрані мови реалізації}

\textbf{Standard ML}\footnote{\url{https://github.com/o1}}. В академічних цілях Марат Хафізов створив
за специфікаєю N2O/NITRO порт на мови Standard ML (SML/NJ та MLton).
Ця робота представлена Github організацією O1 в структурі N2O.

\textbf{Haskell}\footnote{\url{https://github.com/o3}}. Перший експеримент з формалілазції N2O в систему F
була робота Андрія Мельникова. Пізніше, більш повну версію з NITRO протокол
запропонував Марат Хафізов, ця версія представлена на Github як організація O3.

\textbf{F\#}\footnote{\url{https://ws.erp.uno}}. Також у якості вправи Siegmentation Fault зробив порт NITRO
на мову F# разом з ETF кодуванням. Ці напрацювання представлені в Github
організації O61.

\textbf{Lean 4}\footnote{\url{https://github.com/o89}}. У якості більшо формальної платформи з залежними типами,
мова Lean 4 від Леонардо де Мура з Microsoft. Siegmentation Fault автор порта,
який представлений Github організацією O89 та сайтом lean4.dev.

\textbf{Erlang/Elixir}\footnote{\url{https://github.com/synrc}}. Єдина промислова платформа, яка в повній мірі реалізує
специфікацію N2O.TECH --- Erlang/Elixir.

\textbf{Hamler}. Нова формальна платоформа на базі PureScript для віртуальної машини Erlang.
В подальшому цей розділ буде присвячений імплементації та специфікації
на мові Hamler (варіація PureScript з бекендом в Erlang Core).

\begin{table}[h]
\begin{center}
\caption{Перелік мов для яких існує версія базової бібліотеки}
\tabcolsep7pt\begin{tabular}{lcccc}
\hline
\textbf{Мова/Платформа} & \textbf{Набір реалізованих компонент} \\
\hline
Erlang/OTP & N2O, BPE, KVS, NITRO, MAD, FORM \\
Elixir/OTP & N2O, BPE, KVS, NITRO, FORM \\
\hline
Hamler/OTP & RT, BASE, N2O, BPE, KVS \\
\hline
Standard ML & N2O, NITRO, ETF \\
Haskell & N2O, NITRO, ETF \\
F\# & N2O, NITRO, ETF \\
\hline
Lean 4 & N2O, NITRO, MAD, ETF \\
\hline
\end{tabular}
\end{center}
\end{table}

Авторськими вважаються імплементації на мовах Erlang (Elixir) та Hamler (PureScript).
Інші представлені імплементації вважаються сертифікаваними формальними референсними моделями.

Протягом 8 років автор практикувався аби адаптувати бібліотеку середовища виконання
до основних формальних або напівформальних мов System F, які принаймі написані
на мовах які теж є мовами System F (SML, F#, Haskell). Сторонніми дослідниками
була навіть зроблена адаптація для Lean 4\footnote{\url{https://lean4.dev}}.
В процесі цьої багаторічної вправи стало зрозумілим, що досягти на виробництві
тієї якості, яка досягається в середовищі виконання Erlang/OTP майже неможливо.
Тому модель базової бібліотеки, спочатку адаптувалася з оригінальної мови Erlang (2013)
для мови Elixir (2018), і потім для мови Hamler (2021), що є варіацією PureScript (System $F_\omega$).
Всього існує 7 моделей для 7 мов програмування базової бібліотеки представленої в цій роботі.

\section{Пакети формального середовища}
Тут представлена модель бібліотеки формального середовища виконання, яке
скаладається з JIT-інтерпретатора, потужної SMP-системе акторів з неблокуючими
курсорами черг повідомлень для системи процесів 1:1 (процес-черга, кожен процес
має свою чергу). Така спрощена модель дещо відрізняється від моделі CSP, CCS
та класичного числення процесів (Pi Calculus) проте протягом цих 8 років на практиці
стало зрозумілим, що модель Erlang/OTP більш гнучка до масштабування та стійка до помилок.

Іншими словами модель в System F базової бібліотеки
середовища виконання формально визначає цей прошарок уніфікованого мовного середовища.
А модель процесів хоча формально не відображає семантику коіндукції (стерта інформація),
проте досі синтаксис мовного середовища представленого в розділі 2 є сумісним з моделлю
Erlang/OTP яка є основною платформою, що підтримується у виробництві.

\subsection{Структури даних BASE}
Структури даних представлені додатком BASE. Основні модулі додатку:
List, Array, Atomics, Binary, Bool, Char, Counters, DateTime, Digraph,
Enum, Eq, Float, Int, Map, Maybe, Ord, OrdDict, OrdSet, Pid, Queue,
Read, Record, Regex, Set, Time, Tuple.

Сигнатури додатка BASE максимально сумісні з базовою бібліотекою Hamler.

\subsection{Сервіси середовище виконання RT}
Сервіси середовища виконання представлені додатком RT, іменні простори якого
дещо відрізняються від відповідних сигнатур базової бібліотеки мови Hamler.

\subsubsection{Database}
Модулі простору Database: Mnesia, ETS.

\subsubsection{Filesystem}
Модулі простору Filesystem: Dir, File, FilePath, IO, Active.

\subsubsection{OS}
Модулі простору OS: Env, Init, Shell.

\subsubsection{Process}
Модулі простору Process: Application, Supervisor, Dict, GenServer, Spawn, Process, Timer.

\subsubsection{Network}
Модулі простору Network: TCP, UDP, TLS, Inet, WebSocket.

\subsection{Прикладні протоколи N2O}
Сервіси веб-сокет сервера, представлені додатком N2O: N2O, PI, Proto,
Ring, MQTT, WS, TCP, Heart, Syn, FTP, NITRO, ETF, GCM, Session, Cache.

Модуль N2O пропонує ряд сервісів зручних та вивірених в промисловій роботі для побудови
сервісних протоколів що вбудовуються в цикли сучасних TCP, QUIC, UDP, MQTT, WebSocket серверів,
та в процесі своєї роботи можуть стартувати додаткові процеси для обробки інформації під
супервізією середовища виконання.

\begin{lstlisting}
pickle     :: Binary -> Binary
depickle   :: Binary -> Binary
encode     :: forall k. k -> Binary
decode     :: forall k. Binary -> IO k
reg        :: forall k. k -> IO k
unreg      :: forall k. k -> IO k
send       :: forall k v z. k -> v -> IO z
getSession :: forall k v. k -> IO v
putSession :: forall k v. k -> v -> IO v
getCache   :: forall k v. Atom -> k -> IO v
putCache   :: forall k v. Atom -> k -> v -> IO v
\end{lstlisting}

Тут залишено лише саме необхідне, але не настільки тривіальне аби бути іграшковим.
Всі функції всії сервісів можуть підмінятися в ході виконання.
В сутності тут зібрані сервіси:
1) бінарної серіалізації Erlang Term Format (ETF), функції encode/decode;
2) функції симетричного шифрування AES-GCM/128 pickle/depickle;
3) функції Pub/Sub внутрішнього Erlang реєстра (syn/gproc/global, тощо);
4) функції роботи з персональними сесіями, які захищені зашифрованими токенами;
5) функції роботи з глобльним кеш-сервісом для бізнес-об'єктів.

Модуль N2O.PI абстрагує користувача від надмірного
фольклору Supervisor та пропонує простіший протокол запуску
асинхронних процесів.

\begin{lstlisting}
data PI  = PI String Atom Atom Atom Integer RestartType
data Sup = Ok Pid String
         | Error String

startPi :: PI -> IO Sup
\end{lstlisting}

\subsection{Сховище даних KVS}
Сервіси сховища даних представлені додатком KVS: KVS,
Stream, St, Rocks, Mnesia, FS.

\subsection{Бізнес-процеси BPE}
Сервіси системи управління бізнес-процесами
представлені додатком BPE: BPE, Event, Action, Process, Hist, Flow.

\subsection{Контрольні елементи протоколу NITRO}
Сервіси веб-фреймворка, представлені додатком NITRO: NITRO, Combo,
Edit, Form, Input, Table, Actions, Render.

