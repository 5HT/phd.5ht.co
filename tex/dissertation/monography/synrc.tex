\usepackage[osf]{mathpazo}
\usepackage{musicography}
\usepackage{stmaryrd}
\usepackage{color,colortbl}
\usepackage{epigraph}
\usepackage[mathcal,mathbf]{euler}
\usepackage{amsmath,amssymb,amsthm,wasysym}
\usepackage{graphicx,sidecap,tikz}
\usepackage{fullwidth}
\usepackage[T1]{fontspec}
\usepackage{hyphenat}
\usepackage{ifthen}
\usepackage{tikz-cd}
\usepackage[english,russian]{babel}
\usepackage{tabstackengine}
\usepackage{graphicx}
\usepackage{cite}
\usepackage{hyperref}
\usepackage{moreverb}
\usepackage{listings}
\usepackage{caption}
\usepackage{newunicodechar}

\definecolor{keywordcolor}{rgb}{0.0, 0.1, 0.6}
\definecolor{commentcolor}{rgb}{0.4, 0.4, 0.4}   % grey
\definecolor{symbolcolor}{rgb}{0.0, 0.1, 0.6}    % blue
\definecolor{sortcolor}{rgb}{0.1, 0.5, 0.1}      % green

\newunicodechar{≔}{≔}
\newunicodechar{₁}{₁}
\newunicodechar{₂}{₂}
\newunicodechar{₃}{₃}

\usetikzlibrary{matrix}
\usetikzlibrary{babel}
\theoremstyle{definition}
\newtheorem{theorem}{Теорема}
\newtheorem{definition}{Визначення}
\newtheorem{exercise}{Вправа}
\newtheorem{example}{Приклад}

\definecolor{LightGray}{rgb}{0.8,0.8,0.8}
\definecolor{LightGray25}{rgb}{0.9,0.9,0.9}

\newcommand*{\incmap}{\hookrightarrow}

\def\mapright#1{\xrightarrow{{#1}}}
\def\mapleft#1{\xleftarrow{{#1}}}
\def\mapup#1{\Big\uparrow\rlap{\raise2pt{\scriptstyle{#1}}}}
\def\mapupl#1{\Big\uparrow\llap{\raise2pt{\scriptstyle{#1}}}}
\def\mapdown#1{\Big\downarrow\rlap{\raise2pt{\scriptstyle{#1}}}}
\def\mapdownl#1{\Big\downarrow\llap{\raise2pt{\scriptstyle{#1}}}}
\def\mapdiagl#1{\vcenter{\searrow}\rlap{\raise2pt{\scriptstyle{#1}}}}
\def\mapdiagr#1{\vcenter{\swarrow}\rlap{\raise2pt{\scriptstyle{#1}}}}

% To get lining figures in tables set by siunitx, which apparently uses the
% \mathrm font instead of \mathnormal
\SetMathAlphabet{\mathrm}{normal}{U}{eur}{m}{n}

% =========================
% = Setting up the layout =
% =========================

% With a 9pt body font we want a little extra line spacing (I mean *leading*)
\setSingleSpace{1.2}
\SingleSpacing

% Okay, holy crap. Calculating the correct type block height by hand is quite
% challenging (partially because not all lines are \baselineskip high;
% apparently the first line is \topskip high?), and \checkandfixthelayout will
% in the end actually *change* it so that the type block is always an integer
% multiple of lines. The easiest thing is to set the approximate desired type
% block height, the width, the left or right margin, the bottom margin, and
% the headdrop, and then let memoir take care of everything else. Changing the
% algorithm used to check the layout helps as well.
\setstocksize{9in}{6in}
%\stockimperialvo

\settrimmedsize{\stockheight}{\stockwidth}{*}
\settrims{0pt}{0pt}

\settypeblocksize{46\baselineskip}{4in}{*}
\setlrmargins{*}{0.5in}{*}
\setulmargins{*}{0.5in}{*}

\setheadfoot{\baselineskip}{\baselineskip} % headheight and footskip
\setheaderspaces{0.5in}{*}{*} % headdrop, headsep, and ratio

\checkandfixthelayout[lines]

% Set up custom headers and footers
\makepagestyle{stylish}
\copypagestyle{stylish}{headings}
\makerunningwidth{stylish}{5in}
\makeheadposition{stylish}{flushleft}{flushright}{}{}
\pagestyle{stylish}

% ============================
% = Table of contents tweaks =
% ============================
\renewcommand*{\contentsname}{Table of Contents}
\setsecnumdepth{subsubsection}
\settocdepth{subsection}

% ============
% = Chapters =
% ============
\newcommand{\swelledrule}{
    \tikz \filldraw[scale=.015,very thin]
    (0,0) -- (100,1) -- (200,1) -- (300,0) --
    (200,-1) -- (100,-1) -- cycle;}
\makeatletter
\makechapterstyle{grady}{
    \setlength{\beforechapskip}{0pt}
    \renewcommand*{\chapnamefont}{\large\centering\scshape}
%    \renewcommand*{\chapnumfont}{\large}
%    \renewcommand*{\printchapternum}{
%        \chapnumfont \ifanappendix \thechapter \else \numtoName{\c@chapter}\fi}
    \renewcommand*{\printchapternonum}{
        \vphantom{\printchaptername}
        \vphantom{\chapnumfont 1}
        \afterchapternum
        \vskip -\onelineskip}
    \renewcommand*{\chaptitlefont}{\Huge\itshape}
    \renewcommand*{\printchaptertitle}[1]{
        \centering\chaptitlefont ##1\par\swelledrule}}
\makeatother
\chapterstyle{grady}

% See below, after introduction, for \clearforchapter

% Prevent page numbers from appearing on chapter pages
\aliaspagestyle{chapter}{empty}

% ===================
% = Marginal labels =
% ===================
\strictpagecheck % Turn on robust page checking
\captiondelim{} % Don't print a colon after "Figure #.#"

\makeatletter
\renewcommand{\fnum@figure}{
    \checkoddpage
    \ifoddpage
        \makebox[0pt][l]{\hspace{-1in}{\scshape\figurename~\thefigure}}
    \else
        \makebox[0pt][r]{{\scshape\figurename~\thefigure}\hspace*{-5in}}
    \fi
    }

\renewcommand{\fnum@table}{
    \checkoddpage
    \ifoddpage
        \makebox[0.7in][l]{{\tablename ~ \thetable}}
    \else
        \makebox[0.7in][r]{{\tablename ~ \thetable}}
    \fi
    }

\let\mytagform@=\tagform@
\def\tagform@#1{
\checkoddpage
    \ifoddpage
    \makebox[1sp][l]{\hspace{-5in}\maketag@@@{(\ignorespaces#1 \unskip \@@italiccorr)}}
    \else
    \makebox[1sp][r]{\maketag@@@{(\ignorespaces#1
                                  \unskip \@@italiccorr)}\hspace*{-1in}}
    \fi
    }
\renewcommand{\eqref}[1]{\textup{\mytagform@{\ref{#1}}}}
\makeatother

\usetikzlibrary{arrows,positioning,decorations.pathmorphing,trees}

\lefthyphenmin=1
\hyphenpenalty=100
\tolerance=3000

\fontencoding{T1}
\newfontfamily{\cyrillicfont}{Geometria}
\setmainfont{Geometria}

\addto\captionsrussian{\renewcommand{\contentsname}{Зміст}}
\addto\captionsrussian{\renewcommand{\bibname}{Список використаних джерел}}
\addto\captionsrussian{\renewcommand{\chaptername}{Розділ}}
\addto\captionsrussian{\renewcommand{\tablename}{Таблиця}}

\def\lstlanguagefiles{syntax.tex}
\lstset{language=infinity}
