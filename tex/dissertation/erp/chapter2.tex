\chapter{Загальна структура}

        У цій статті я розповім, з чого складається модуль підприємства.

        Якщо розглядати первинну гранулярність підприємства, яке
           в першому наближенні складається з модулів, то уже на прикладі PLM
           ми можемо виявити повну структуру типового модуля, кожен елемент
           якої представлений в екземплярі PLM.

        \section{Конфігурація}

        Перший і головний компонент додатка — файл конфігурації
           (для Erlang — \textbf{sys.config}, для Elixir — \textbf{consig.exs}),
           який потрібен для багатьох додатків-залежностей:
           \textbf{n2o}, \textbf{kvs}, \textbf{erp}, \textbf{form}.
           Це обов'язковий компонент будь-якого ерланг додатка, який
           потребує ці залежності.

        Більш детально про конфігурацію Erlang і Elixir додатків можна почитати тут:

        — <a href="https://erlang.org">Erlang</a><br />
           — <a href="https://elixir-lang.org">Elixir</a>

        \section{Публикація}

        Для побудова релізу, звичайного запуску або публікації в hex.pm
           за допомогою \textbf{mad}, \textbf{mix} чи \textbf{rebar3},
           вам необхідний файл публікації (для Erlang — \textbf{rebar.config},
           для Elixir — \textbf{mix.exs}). Файл публікації містить план запуску додатків.

        Більш докладно про публікацію Erlang і Elixir додатків можна почитати тут:

        — <a href="https://mad.n2o.space">mad</a> (Erlang)<br />
           — <a href="https://www.rebar3.org">rebar3</a> (Erlang)<br />
           — <a href="https://elixir-lang.org/getting-started/mix-otp/introduction-to-mix.html">mix</a> (Elixir)

        \section{Типові специфікації}

        Типова специфікація — це сукупність визначень типів (type),
           записів (record) та специфікацій функцій (spec). Це информація для діалайзера,
           який допомагає визначити невідповідність коду цим специфікаціям. Всі системи
           збірки підтримують перевірку dialyzer.

        У типових специфікаціях ми зберігаємо внутрішні структури
           фреймворків і додатків, а також бізнес-об'єктів підприємства.
           Мова опису бізнес об'єктів підтримує кортежі (для повідомлень),
           суми (для протоколів), скалярні і векторні типи (для полів).

        Типові специфікації зберігаються в HRL файлах, в папці \textbf{include}.
           Тут повинні бути всі -spec, -record, -type визначення. В Elixir
           імпортуйте їх за допомогою \textbf{Record.extract}.

        Якщо програма не містить include папки (наприклад, як PLM модуль),
           то це означає, що модуль не визначає ніяких додаткових типів,
           а користується типами своїх залежностей, або не користується ними взагалі.

        Більш докладно про типові специфікації та підтримувані мови програмування
           можна почитати тут:

        — <a href="https://bert.n2o.space">bert</a>

        \section{Протоколи}

        Якщо додаток реалізує якийсь протокол, цей протокол вбудовується
           в протокольні цикли n2o_mqtt, n2o_ws, n2o_tcp розподіленого кільця воркерів,
           які обслуговують запити клієнтських додатків.

        Список протоколів визначається у змінній protocols бібліотеки N2O:

        \begin{lstlisting}
  protocols:
  [
    :n2o_heart,
    :n2o_nitro,
    :n2o_ftp,
    :bpe_n2o,
    CHAT.TXT
  ]
        \end{lstlisting}

        А список воркерів, які реалізують ці протоколи — на ендпоінтах:

        \begin{lstlisting}
  mqtt_services: ['erp', 'plm'],
  ws_services: ['chat'],
        \end{lstlisting}

        Протоколи, якщо вони реалізовані додатком,
           знаходяться в папках \textbf{src/protos} і \textbf{lib/protos}
           для Erlang і Elixir відповідно.

        Більш докладно про N2O протоколи та їх використання
           можно почитати тут:

        — <a href="https://ws.n2o.space">n2o</a>

        \section{Ланцюжки}

        Все типізовані типовими специфікаціями дані зберігаються в KVS сховищі.
           Це Erlang-орієнтована абстракція над
           записами/кортежами (records, tuples, C-structures), яка
           дозволяє приховувати за єдиним інтерфейсом декілька KV сховищ (включаючи
           Mnesia, RocksDB, Cassandra).

        \section{Кортежі і їх ланцюжки}

        В основі KVS лежать поняття кортежа і ланцюжка. Типи кортежів визначаються
           в типових специфікаціях, а ланцюжки — це послідовності кортежів,
           в загальному випадку будь-яких типів, таким чином можна говорити про гетерогенні та гомогенні ланцюжки.

        Кожен ланцюжок індексується своїм ідентифікатором, який представляє
           собою сегментований шлях в ієрархічній віртуальній файловій системі.
           Це зроблено для того, щоб префіксним пошуком можна було вибрати всіх
           дітей певної субгілки в ієрархії ідентифікаторів ланцюжків. Всі ідентифікатори
           всіх ланцюжків також знаходяться в ланцюжку.

        \section{Схеми}

        Кожен модуль підприємства може включати одну чи багато схем.
           Схема — це сукупність типових специфікацій, іншими словами — певний набір
           кортежів і їх типів, як форма дистрибуції
           типової специфікації.

        \section{Первинні кореневі ланцюжки}

        Щоб не створювати руками всі базові словники і основні організаційні структури,
           зручно винести їх в так звані завантажувальні модулі. Ці записи автоматично
           створюються при холодному старті додатку \textbf{ERP}. Кореневим первинним ланцюжкам
           присвячені одразу дві наступні частини:
           Частина 3. Створення первинних ланцюжків, де розповідається, як створювати організаційну
           структуру підприємства у вигляді завантажувальних модулів первинних кореневих ланцюжків. і
           Частина 4. Створення адміністратора даних, де розповідається, як створити
           універсальний переглядач ланцюжків у вигляді окремого модуля підприємства.

        Більш докладно про систему зберігання KVS та управління типовими специфікаціями
            можна почитати тут:

        — <a href="https://kvs.n2o.space">kvs</a>

        \section{Процеси}

        Якщо всі дані інформаційної системи підприємства зберігаються в ланцюжках,
           то еволюція цих даних відбувається за допомогою бізнес-процесів.
           Бізнес-процеси покликані вирішити певні проблеми, пов'язані
           з масштабуванням бізнес-логіки на виробництві, тому ця частина
           підприємства добре стандартизована з 2008 року, з появою більш-менш
           універсального стандарту BPMN, який частково підтримується системою
           управління процесами BPE.

        У загальному випадку бізнес процеси (БП) - це графові уявлення алгоритму з іменами переходів,
           станів і асоційованих функцій. Всі бізнес-модулі підприємства реалізують
           якийсь головний БП, і серію допоміжних процесів. Покликанням БП є вирішення проблеми
           ізоляції розподіленої транзакції у вигляді окремого процесу віртуальної машини.
           Цей БП являє собою звичайну функцію \textbf{action/2}, аргументами якої
           є ідентифікатори ланцюжків. У якості ефектів цей БП генерує дані
           в інших ланцюжках, реалізуючи таким чином обчислювальну модель обчислення процесів.

        Наприклад, БП "Рахунок в Банку" є циклічним рекурентним процесом,
           який виходить з, і входить в один і той же стан (моноїд). В якості аргумента
           у цієї функції, що складається з однієї умови, є тільки скалярна
           величина — бізнес об'єкт "Транзакція". Таким чином, трейс цього
           процесу — ланцюжок транзакцій. Операція переказу грошей в такій моделі
           означатиме розподілену транзакцію між усіма учасниками переказу,
           контрольовану окремим процесом.

        У системі, яку розглядаємо, модуль PLM включає три процеси:
           1) Процес "Рахунок в Банку" фінансового модуля FIN;
           2) Процес "Продукт" модуля PLM;
           3) Процес "Пре-Продукт" модуля PLM.
           В Частині 5. Адміністратор процесів показано, як створити адміністратора процесів
           для модуля BPE, призначеного для ознайомлення з системою, а також для
           примітивного ручного тестування.

        Більш докладно про систему управління бізнес-процесами BPE та її використання
           можна почитати тут:

        — <a href="https://bpe.n2o.space">bpe</a>

        \section{Сторінки}
        \section{Редактори}

        <figure><img src="../images/15.png" /></figure>

        \section{Вектори}
        \section{Роутери}

        Більш докладно про веб-фреймворк NITRO
           можна почитати тут:


\section{}
