\chapter{Створення ланцюжків}

        У цій статті ми покажемо, як визначати схеми бізнес-об'єктів
           підприємства та ієрархічну структуру ідентифікаторів ланцюжків
           цих об'єктів. Це визначає консольну взаємодію користувача
           та інформаційної системи.

    </section>
    <section>


        \section{Типові специфікації}

        Схема включає у себе всю типову інформацію про бізнес-об'єкти.
           Схема ERP модуля визначає базові об'єкти організаційної структури,
           об'єкти платіжної системи PAY та систему обліку співробітників ACC.

        <h4>ERP SPEC — Таблиці организаційної структури</h4>

   \begin{lstlisting}
-type locationType() :: normal | extra.

-record('Loc',
{ id          = kvs:seq([],[]) :: [] | term(),
  code        = [] :: [] | term(),
  country     = [] :: [] | binary(),
  city        = [] :: [] | binary(),
  address     = [] :: [] | binary(),
  type        = [] :: locationType() }).

-record('Branch',
{ id          = kvs:seq([],[]) :: [] | term(),
  loc         = [] :: [] | #'Loc'{} }).

-record('Inventory',
{ id          = [] :: [] | binary(),
  name        = [] :: [] | binary(),
  branch      = [] :: [] | #'Branch'{},
  type        = [] :: term() }).

-record('Organization',
{ name        = [] :: [] | binary(),
  url         = [] :: [] | string(),
  location    = [] :: [] | #'Loc'{},
  type        = [] :: term() }).

-record('Person',
{ id          = kvs:seq([],[]) :: [] | term(),
  cn          = [] :: [] | binary(),
  name        = [] :: [] | binary(),
  displayName = [] :: [] | binary(),
  location    = [] :: #'Loc'{},
  type        = [] :: term() }).


-record('Employee',
{ id          = kvs:seq([],[]) :: [] | binary(),
  person      = [] :: [] | #'Person'{},
  org         = [] :: [] | #'Organization'{},
  branch      = [] :: [] | #'Branch'{},
  type        = [] :: term() }).
    \end{lstlisting}

        <h4>PAY SPEC — Таблиці системи управління платежами</h4>

        Гроші зберігаються у форматі \textbf{{N,M}}, де \textbf{N} — кількість знаків після коми,
           а \textbf{M} — усі значущі цифри. Таким чином числа кодуються багатьма способами,
           наприклад одиниця: 1 = {0,1} = {1,10} = {2,100}. Операція множення у такій системі
           виглядає просто \textbf{mul({A,B},{C,D}) -> {A+C,B*D}}.
           

   \begin{lstlisting}
-type fraction_length() :: integer().
-type digits() :: integer().
-type money() :: {fraction_length(),digits()}.

-record('Payment',
{ invoice = [] :: [] | term(),
  volume = [] :: [] | money(),
  price = {0,1} :: money(),
  instrument = [] :: term(),
  type = [] :: paymentType(),
  from = [] :: term(),
  to = [] :: term() }).
    \end{lstlisting}

        <h4>PLM SPEC — Таблиці системи управління життєвим циклом</h4>

   \begin{lstlisting}
-record('Acc',
{ id   = [] :: [] | binary() | list(),
  rate = {0,0} :: money() }).

-record('Product',
{ code         = [] :: [] | term(),
  id           = kvs:seq([],[]) :: [] | binary(),
  url          = [] :: [] | binary() | list(),
  engineer     = [] :: [] | #'Person'{},
  director     = [] :: [] | #'Person'{},
  owner        = [] :: [] | #'Person'{},
  organization = [] :: [] | #'Organization'{},
  type         = [] :: productType() }).

-record('Investment',
{ id = [] :: [] | term(),
  volume = [] :: [] | money(),
  price = {0,1} :: money(),
  instrument = [] :: term(),
  type = [] :: investmentType(),
  from = [] :: term(),
  to = [] :: term() }).
    \end{lstlisting}

    </section>
    <section>
        \section{Створення кореневих ланцюжків}

        ERP BOOT або пусконаладка підприємства — це процес заповнення первинних
           словників та таблиць фундаментальною інформацією. Головним чином це відображення
           ієрархічної, організаційної структури підприємства.
           Від працівника підприємства, його рабочого місця,
           його локального офісу, його локальної компанії, і далі, до
           группи міжнародних компанії з офісами в різних країнах світу,
           та можливо, навіть до синдикатів транснаціональних корпорацій.

        <h4>ERP BOOT — Організаційна структура підприємства</h4>

        Розглянемо приклад: компанія Quanterall, головний підрядник Aeternity, має офіси в
           Софії, Варні (головний офіс компанії) та Пловдіві. Сама компанія здійснює операції
           тільки в Болгарії, тому группа складається из однієї компанії.

        Добавляємо, зараз та надалі, дані за допомогою list комбінаторів:

   \begin{lstlisting}
-module(erp).
-compile(export_all).

boot() ->
    GroupOrgs =
  [ #'Organization'{
      name="Quanterall",
      url="quanterall.com"} ],

    HeadBranches =
  [ #'Branch'{ loc = #'Loc'{ city =
     "Varna", country = "BG" } },
    #'Branch'{ loc = #'Loc'{ city =
     "Sophia", country = "BG" } },
    #'Branch'{ loc = #'Loc'{ city =
     "Plovdiv", country = "BG" } } ],

    PartnersOrgs =
  [ #'Organization'{ name="NYNJA"},
    #'Organization'{ name="Catalx"},
    #'Organization'{ name="FiaTech"},
    #'Organization'{ name="3Stars"},
    #'Organization'{ name="SwissEMX"},
    #'Organization'{ name="HistoricalPark"},
    #'Organization'{ name="Intralinks"} ],

    Structure =
  [ {"/erp/group", GroupOrgs},
    {"/erp/partners", PartnersOrgs},
    {"/erp/quanterall", HeadBranches} ],

    lists:foreach(fun({Feed, Data}) ->
       case kvs:get(writer, Feed) of
            {ok,_} -> skip;
         {error,_} -> lists:map(fun(X) ->
                      kvs:append(X,Feed)
       end, Data) end end, Structure).
    \end{lstlisting}

        <h4>PAY BOOT — Звітність CashFlow</h4>

        Управління звітністю аутсорс підприємства доволі просте:
           1) ми приймаємо оплати по інвойсам, виставленим клієнтам, з періодом раз на місяць;
           2) ми виплачуємо зарплати раз на місяць. Тому згортки групуються календарно
           та зіпуються помісячно.

   \begin{lstlisting}
sal_boot() ->
   lists:map(fun(#'Product'{code=C} = P) ->
      lists:map(fun(#'Payment'{}=Pay) ->
         kvs:append(Pay,
         "/plm/"++C++"/outcome") end,
           salaries(C))
      end, products()).

pay_boot() ->
   lists:map(fun(#'Product'{code=C} = P) ->
      lists:map(fun(#'Payment'{}=Pay) ->
         kvs:append(Pay,
         "/plm/"++C++"/income") end,
           payments(C))
      end, products()).
    \end{lstlisting}

        <h4>PLM BOOT — Бюджетування проектів</h4>

       Іниціалізація погодинна для кожного співробітника по проекту.
          Цей список буде використовуватись в майбутньому для розподілення
          виплат по опціонах.\textbf{} 

   \begin{lstlisting}
assignees() ->
   lists:map(fun(#'Product'{code=C} = P) ->
      case kvs:get(writer,"/plm/"++C++"/staff") of
           {error,_} ->
             lists:map(fun(#'Person'{}=Person) ->
             kvs:append(Person,
             "/plm/"++C++"/staff") end,staff(C));
           {ok,_} -> skip end
      end, products()).
    \end{lstlisting}


    Виплати за процентами на субконто попроектно:

   \begin{lstlisting}
accounts() ->
  lists:map(fun(#'Product'{code=C}) ->
    lists:map(fun(#'Acc'{id=Id, rate=R}=SubAcc) ->
      Address = lists:concat(["/fin/acc/",C]),
      kvs:append(SubAcc,Address),
      Feed = lists:concat(["/fin/tx/",Id]),
      case kvs:get(writer, Feed) of
           {error,_} ->
             lists:map(fun(#'Payment'{
               invoice=I,price=P, volume=V}=Pay) ->
             kvs:append(rate(Pay,SubAcc,C), Feed)
             end, payments(C));
             {ok,_} -> skip
      end
    end, acc(C))
  end, plm_boot:products()).
    \end{lstlisting}

        За работу з даними відповідає біблиотека KVS, як працювати з нею читайте в минулих
           випусках журналу:
          — <a href="https://tonpa.guru/stream/2019/2019-04-13%20Новая%20версия%20KVX.htm">2019-04-13 Нова Версія KVS</a><br />
          — <a href="https://github.com/synrc/kvs">synrc/kvs</a>
    </section>
    <section>
        \section{Інкапсуляція структури підприємства}

        Увесь код, який потрібен для створення фідів, зазвичай виносять в додаток
           з назвою ERP. Для кожного конкретного підприємства ми використовуємо свою
           Github організацію, можна навіть інше ім'я репозиторію, але ім'я
           Erlang/OTP додатку — завжди залишається ERP:

          — <a href="https://github.com/enterprizing/erp">enterprizing/erp</a>

        У цьому репозиторії завжди будуть якісь реальні дані якоїсь компанії,
           яку ми автоматизуємо, якщо вони дозволять нам опублікувати свою
           організаційну структуру, первинні дані та словники. В інших випадках цей репозиторій
           робимо приватним. Самі додатки здатні працювати з довільними структурами ERP.
    </section>
    <section>
        \section{Приклади запитів до сховища}

        Elixir прелюдія:

   \begin{lstlisting}
defmodule PLM.Mixfile do
  use Mix.Project
  def project() do
    [
      app: :plm,
      version: "0.7.1",
      elixir: "~> 1.8.1",
      description: "PLM Product Lifecycle Management",
      deps: [{:bpe, "~> 4.7.3"}, {:erp, "~> 0.7.6"}]
    ]
  end
  def application(),
    do: [mod: {PLM.Application, []},
         applications: [:rocksdb, :kvs, :bpe, :erp]]
end
    \end{lstlisting}

        Зміст кореневої директорії БД підприємства:

   \begin{lstlisting}
>  :writer |> :kvs.all |> :lists.sort
[
  {:writer, '/acc/quanterall/Plovdiv', 3, [], [], []},
  {:writer, '/acc/quanterall/Sophia', 9, [], [], []},
  {:writer, '/acc/quanterall/Varna', 23, [], [], []},
  {:writer, '/bpe/hist/1562855060639704000', 1, [], [], []},
  {:writer, '/bpe/proc', 1, [], [], []},
  {:writer, '/erp/group', 1, [], [], []},
  {:writer, '/erp/partners', 7, [], [], []},
  {:writer, '/erp/quanterall', 3, [], [], []},
  {:writer, '/fin/acc/CATALX', 4, [], [], []},
  {:writer, '/fin/acc/NYNJA', 4, [], [], []},
  {:writer, '/fin/tx/CATALX/R&#38;D', 12, [], [], []},
  {:writer, '/fin/tx/CATALX/insurance', 12, [], [], []},
  {:writer, '/fin/tx/CATALX/options', 12, [], [], []},
  {:writer, '/fin/tx/CATALX/reserved', 12, [], [], []},
  {:writer, '/fin/tx/NYNJA/R&#38;D', 5, [], [], []},
  {:writer, '/fin/tx/NYNJA/insurance', 5, [], [], []},
  {:writer, '/fin/tx/NYNJA/options', 5, [], [], []},
  {:writer, '/fin/tx/NYNJA/reserved', 5, [], [], []},
  {:writer, '/plm/CATALX/income', 12, [], [], []},
  {:writer, '/plm/CATALX/investments', 4, [], [], []},
  {:writer, '/plm/CATALX/outcome', 12, [], [], []},
  {:writer, '/plm/CATALX/staff', 2, [], [], []},
  {:writer, '/plm/NYNJA/income', 5, [], [], []},
  {:writer, '/plm/NYNJA/investments', 2, [], [], []},
  {:writer, '/plm/NYNJA/outcome', 5, [], [], []},
  {:writer, '/plm/NYNJA/staff', 4, [], [], []},
  {:writer, '/plm/products', 2, [], [], []}
]
    \end{lstlisting}

        Список компаній, які входять в групу підприємства:

   \begin{lstlisting}
> :kvs.feed '/erp/group'
[{:Organization, 'Quanterall', 'quanterall.com', [], []}]
    \end{lstlisting}

        Список бренч-офісів головної (та єдиної) компанії групи:

   \begin{lstlisting}
> :kvs.feed '/erp/quanterall'
[
  {:Branch, '1562329445378242000',
   {:Loc, '1562329445378243000', [], 'BG', 'Plovdiv', [], []}},
  {:Branch, '1562329445378241000',
   {:Loc, '1562329445378242000', [], 'BG', 'Sophia', [], []}},
  {:Branch, '1562329445378234000',
   {:Loc, '1562329445378240000', [], 'BG', 'Varna', [], []}}
]
    \end{lstlisting}

        Список контрагентнів:

   \begin{lstlisting}
> :kvs.feed '/erp/partners'
[
  {:Organization, 'Catalx Exchange Inc.', 'catalx.io', [], []},
  {:Organization, 'HistoricalPark', [], [], []},
  {:Organization, 'NYNJA, Inc.', 'nynja.io', [], []},
  {:Organization, 'Intralinks', [], [], []},
  {:Organization, 'SwissEMX', [], [], []},
  {:Organization, 'FiaTech', [], [], []},
  {:Organization, '3Stars', [], [], []}
]
    \end{lstlisting}

    Бюджетування проекту по статтям субконто:

   \begin{lstlisting}
> :kvs.feed '/fin/acc/NYNJA'
[
  {:Acc, 'NYNJA/insurance', {2, 70}},
  {:Acc, 'NYNJA/reserved', {2, 10}},
  {:Acc, 'NYNJA/options', {2, 10}},
  {:Acc, 'NYNJA/R&#38;D', {2, 10}}
]
    \end{lstlisting}

    Виплати по опціонам для програмістів:

   \begin{lstlisting}
> :kvs.feed '/fin/tx/CATALX/options'
[
  {:Payment, '1562868880497278000', {0, 1}, {2, 150000}, 'USD', :crypto, [], []},
  {:Payment, '1562868880496849000', {0, 1}, {2, 100000}, 'USD', :crypto, [], []},
  {:Payment, '1562868880496409000', {0, 1}, {2, 120000}, 'USD', :crypto, [], []},
  {:Payment, '1562868880495897000', {0, 1}, {2, 150000}, 'USD', :crypto, [], []},
  {:Payment, '1562868880495412000', {0, 1}, {2, 100000}, 'USD', :crypto, [], []},
  {:Payment, '1562868880494920000', {0, 1}, {2, 100000}, 'USD', :crypto, [], []},
  {:Payment, '1562868880494538000', {0, 1}, {2, 100000}, 'USD', :crypto, [], []},
  {:Payment, '1562868880494072000', {0, 1}, {2, 50000}, 'USD', :crypto, [], []},
  {:Payment, '1562868880493672000', {0, 1}, {2, 70000}, 'USD', :crypto, [], []},
  {:Payment, '1562868880493387000', {0, 1}, {2, 150000}, 'USD', :crypto, [], []},
  {:Payment, '1562868880492939000', {0, 1}, {2, 120000}, 'USD', :crypto, [], []},
  {:Payment, '1562868880492234000', {0, 1}, {2, 120000}, 'USD', :crypto, [], []}
]
    \end{lstlisting}

    Люди, які працюють на проекті:

   \begin{lstlisting}
> :kvs.feed '/plm/NYNJA/staff'
[
  {:Person, '1562868880467887000', 'Maxim Sokhatsky', [], [], [], 1, []},
  {:Person, '1562868880467886000', 'Yuri Maslovsky', [], [], [], 8, []},
  {:Person, '1562868880467885000', 'Nikolay Dimitrov', [], [], [], 4, []},
  {:Person, '1562868880467884000', 'Radostin Dimitrov', [], [], [], 4, []},
  {:Person, '1562868880467883000', 'Georgi Spasov', [], [], [], 8, []}
]
    \end{lstlisting}

